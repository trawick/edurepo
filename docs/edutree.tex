\documentstyle[11pt]{article}
\begin{document}
\title{Multi-representation course description}
\author{Jeff Trawick\\Raleigh, NC}
\maketitle

\section*{Basic ideas}

The structure or description of a particular class can be represented in several
different ways:

\begin{itemize}
\item course standards/objectives
\item ``I Can'' lists
\item class syllabus and schedule
\item etc.
\end{itemize}

To a great extent these are different views of the same ideas, though
with different creators and consumers.  Parents and students consume the
class syllabus and schedule and ``I Can'' lists, but most likely will
not understand high level standards and objectives.  Teachers consume 
the course standard/objectives and ``I Can'' lists but create the class
syllabus and schedule for a specific offering of the class.  
The higher level information is shared by
all teachers of a given course within a state (or more) and is created by
domain experts.

Consider a web site that provides a representation
of the structure of the course with the ability for different audiences
to view information appropriate to them.  Because much of the data is
shared among a large number, possibly many thousands, of teachers/class offerings, it is feasible
to provide better information about objectives for students and parents.  (Parents may
currently observe a vast gulf between abstract course standards
maintained by state boards of education and the minimal course calendar
provided by teachers to students, and be unable to properly assist their
child without input from the teacher.  And this input is often needed
in real time, in the context of a particular assignment.)

Here's a description in terms of a particular user interface:

The teacher provides students and parents with a URL representing
a particular course (e.g., 8th period common core 1 math for 9th grade).
View the URL and you see essentially a syllabus and class schedule,
modified by the teacher as necessary and filled in with more information
(e.g., test dates) during the term.

The student or parent can see an abbreviated form of the educational objective
on the schedule (or calendar)
and follow that to common resources such as ``I Can'' statements
or even more abstract descriptions.  Creating this reference, or link, to objectives,
which in turn provides access to shared information, should not require much effort 
on the part of individual teachers and could make it worthwhile for a relatively small
group of people to develop and maintain high quality, common resources.

This representation of the structure of the course is amenable to providing access to other
materials as well.  For teachers and mentors or supervisors, it may be appropriate to 
store and access lesson plans using this structure.  For all parties,
the shared representation of the course could be a gateway to a curated index
to appropriate materials available on the Internet.  More specifically, for a particular objective,
some classes of users should be able to add pointers to resources elsewhere on the 
Internet --- videos, worksheets, interactive activities, etc. --- and
possibly different classes of users may be able to indicate which resources
were particularly helpful (e.g., by ``up voting'' the most pertinent or
highest quality materials).

Beyond the ``free'' materials on the Internet, there are also a lot of
for-fee materials (see http://www.teacherspayteachers.com/).  An opt-in
view could show materials that aren't free.

The representation and presentation of these additional materials is of
course very important, and having an appropriate structure which separates them
by learning style and/or type of material is critical.  For teachers specifically,
the structure can serve as a prompt, or reminder, of the best pedagogical practices.

\section*{Implementation}

The description so far has implied that there is a web site where domain
experts create course outlines and supply information at different levels 
of abstraction, and where teachers can create a view of this specific to
a particular class and add their syllabus/course calendar and tie it all
together.

A more likely scenario is that the common information maintained by domain
experts follows a model for such course descriptions and is stored in a
repository (see separate document).  As the repository contains an API
for accessing the information, web sites could reuse the common data for
different purposes.  E.g., something like Blackboard where a teacher maintains
schedules and assignments and points to different resources could consume
the common information via API for integration into that environment.
Existing web sites which point to educational materials by subject could
reuse the course structure via API instead of inventing a new one.  But
note that enabling the consolidation onto fewer, higher quality indexes of
materials should be a goal of this effort.  This is an example of how the
data could be reused in existing contexts to avoid duplication.

The class structure and common information will be stored in a common
repository for consumption in multiple contexts, but a reference view (web
site) should be provided, with the ability for teachers without existing
learning environments to create class-specific information which integrates
with common information about the subject.

\section*{Existing, partial implementations}

But we already have this, right?  Isn't this Blackboard (or your favorite
web site/software)?

\begin{itemize}
\item A web site which indexes materials for math teachers may represent the structure
of the course in terms of the abstract objectives, and for each objective point
to a few resources that they have curated.
\item Blackboard can be used to create a lot of this but (wild assertion) the
material is generally limited to that which is authored or curated by the
individual teacher.
\item We don't (wild assertion) have a way for individuals from the large population to maintain and curate
an index to the many educational resources available.  (We have lots of indexes that
necessarily recreate representations of class structure of varying quality.)
\item Almost no states/districts/schools (wild assertion) have a way
for students and parents to follow a particular class (syllabus/schedule) and
from there easily access appropriate materials beyond the small amount that the
teacher might have had time to document.  (It isn't simply a matter of integration
for ease of use; the teacher must have a way to easily reference the objective in 
a manner that identifies the relevant resources.)
\end{itemize}

\end{document}
