\documentstyle[11pt]{article}
\begin{document}
\title{Educational Material Repository}
\author{Jeff Trawick\\Raleigh, NC}
\maketitle
A vast amount of educational materials are available currently, both from
educational vendors as well as from a large number of
organizations and individuals.  Much of the material hard to find,
is available under varying terms, and is not easily consumable by teachers,
students, and parents.  Concentrating efforts and standardizing the format
of and access to these materials would be of great benefit to all parties.

This is a brief description of a (theoretical) common repository of educational
materials which are available for use by teachers, students, and parents in 
different contexts, and consumable using different software.  The repository contains
the material itself rather than serve as an index to materials available elsewhere.

\section*{Key characteristics}

The key characteristics of the repository are

\begin{itemize}
\item Materials are machine readable, with a structure and open access mechanism
that allows them to be consumed in varied and unpredictable ways, allowing
maximum reuse of materials using a variety of software.
\item The repository provides some means to consume the materials directly,
without the use of additional software.
\item Materials are created and modifiable in a manner that supports reviewed
collaboration over the Internet with minimal expertise required in the software
which facilitates collaboration.
\item Materials in the repository are all available under a common, permissive
license.
\item The repository contains enough materials and is usable enough to attract
the critical mass of users, both individual and institutional, required to 
grow and improve the contents.
\end{itemize}

(There are a number of characteristics of the organization of the materials,
including the ability of educators to create custom aggregations of the materials, which
are required for the repository to be useful; organizational characteristics
are largely ignored in this brief description.)

\section*{Accessible, machine readable materials}

First, it is useful to give a few examples of different types of materials
from different fields of study:

\begin{itemize}
\item Glossary items (terms and definitions)
\item Examples and counterexamples of sentences with subject-verb agreement
\item Mathematical word problems with corresponding equations and solutions
\item Essay templates with prompts and structure which remind students of the
requirements of a particular type of essay
\end{itemize}

(There are of course many different types of such materials, some which
follow a predictable model, or schema, and some which do not.  The repository
necessarily contains materials that follow a model which has specifically
been designed to represent a particular type of material.  Many types
of materials are out of scope for the repository.)

Such materials can be consumed in many different ways.  Consider the use of
glossary terms in a single class:

\begin{itemize}
\item A biology teacher enters the terms and definitions corresponding
to the current unit of study into a web service such as Quizlet so that
students have access to flash cards and games that help them learn the 
material.
\item A biology teacher prepares a quiz in Microsoft Word by re-typing
or cutting and pasting terms and definitions into several different types of
questions (e.g., matching, pick the correct statement, fill in the blank).
\item A biology teacher prepares a presentation which has a mixture of
different types of materials, including glossary items.
\end{itemize}

(and so on and so forth)

The point is that a particular physical copy of data is typically tied
to the specific software that is used to consume and/or format it, leading
to a large amount of wasted effort.  The inability to reuse the data for
different purposes or with different software is one contributor to the duplication of effort
not just for a single teacher using the material in different contexts but
more importantly among
the many thousands of teachers who are teaching the same subject.

The repository must provide APIs for referencing the materials that
can be exploited by different software.  For example, the biology teacher
in the example above should be able to create a customized selection of
glossary terms for the current unit (or more likely,
review and reuse an existing selection), assign an id to that selection, and
plug that id into a service like Quizlet or other software to quickly reuse
the definitive material.  In short, consume from multiple software
types and implementations without duplicate effort.  (And even the initial
effort should be very small once any teacher using the system has set
up the glossary items used for the class.)

\section*{Direct consumption}

While a key characteristic of the repository is that the materials are
consumable from any educational software (once the APIs are exploited by
the software provider), it is also important to provide
basic capabilities with the repository to make the materials accessible
to users via a web browser.  A reference view of the materials is required,
and other software won't be able to consume the materials immediatel anyway.
But the expectation for the future is that the best quality end user software that is able
to consume the materials will come from other sources,
and that the critical mass of raw materials in the repository will be a
strong incentive for vendors or individuals to experiment with many different
ways to consume it.  The range of possibilities is wide --- web sites, tablet
applications, Microsoft Office plugins, custom devices, etc.

\section*{Collaboration}

Existing models exist for collaboration over the Internet from individuals
all over the world.  Well known examples of this are Wikipedia and the
various (major) open source software projects.  Each group defines the 
manner in which untrusted people can contribute additions or improvements to
the work, as well as the manner in which some contributors are empowered to
make larger changes and help manage the information (i.e., become trusted).
Collaboration has
key technical considerations, particularly the software and procedures which
are used by contributors in order to share information.  Technical considerations
which might be appropriate for developers of software source code would not
be so for educational materials, as low technical barriers are required.

Collaboration also has social considerations.  In general, a lot can be said
about the necessary technical and social aspects of collaboration, but for
now it is sufficient to trust that one of the well known existing models will
be workable.

\section*{Licensing}

The exact licensing terms, including attribution requirements, are key.  All materials in the
repository must be available under the same terms, and those terms should be as simple
and permissive as possible.  Individuals who contribute material or corrections or other 
improvements must acknowledge agreement with the common licensing and that
they are legally entitled to make the contribution (i.e., the material is
their own work and the individual is not covered by an employer agreement
which forbids publishing work in that manner).

One of the unavoidable outcomes of simple attribution requirements is that
the names of individual contributors
will not be associated directly with the material at the point of consumption.
However, tracking of creation and modifications to the materials should allow 
authorship to be obtained from the repository and the contributions of any
particular individual to be reviewed.

An important choice in licensing is whether or not the materials can be reused
for profit.  Some individual contributors may not want their effort to be 
reused by others for profit.  At the same time, the ability to use the 
repository as part of a business will motivate some vendors to assist with
the maintenance or contents of the repository, and APIs can be designed with
constraints that ensure that infrastructure costs are covered by heavy
users of the repository (i.e., not individual accesses).

\section*{Bootstrapping}

The repository must provide value before it will attract a large number
of users, and large number of users must be attracted before volunteers
will surface who will add to and improve the repository.

Among the many resources already available on the Internet, some will
contain large amounts of materials in reasonable formats, and it may be
possible technically and legally to import such materials into the
respository.  It is expected that such existing resources will have a
relatively small scope (e.g., high school mathematics).

Another potential bootstrapping mechanism is to start with a very limited
scope (e.g., only several common core classes) with a specific plan in place
to develop a useful set of materials for those classes.

\section*{Usefulness}

It is natural to question the usefulness of the repository of materials
given the overall scope of teaching and learning.  After all, these materials are
not the essence of teaching and learning, and teachers and schools have
been managing to purchase, create, or otherwise obtain such materials all
along.  The potential benefits are straightforward:

\begin{itemize}
\item Reduce duplicate effort (allow teachers to spend time on more valuable activities).
\item Increase the quality of the materials used (no mispellings, grammatical errors,
or bad examples in crowd-sourced, crowd-reviewed materials).
\item Concentrate efforts on a single repository instead of thousands of
existing web sites with little or no collaboration capabilities (avoiding the
tedius search for existing, freely available materials of sufficient quality).
\item Provide machine readable access to quality materials in order to
improve the usefulness of existing and future software, and in particular
reduce the effort required to experiment with new ideas for consuming the
information.

\end{itemize}

\section*{Possible next steps}

\begin{itemize}

\item Refine the description of the repository so that more people can
understand the key points and help assess its potential value.
\item Survey existing resources to identify best practices and determine
how close existing resources come to meeting requirements.  For example,
``Learning Objects'' and ``Sharable Content Objects'' are existing ideas
that have some intersection with the repository concept.
\item Create a simple prototype to illustrate potential mechanisms for 
collaborating on the creation and maintenance of the materials as well as
consumption of the materials by teachers and students.  Use the prototype
to discuss the potential value of the respository witho a wider audience
and identify particular technical and social challenges.

\end{itemize}

\end{document}
